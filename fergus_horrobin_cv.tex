\documentclass[line, margin]{res}

\begin{document}
%\hspace*{\fill}
\name{Fergus Horrobin\hspace{3.65in} Curriculum Vitae}
%\hspace*{\fill}

\linespread{1.5}

\begin{resume}
\begin{tabular}{l l}
 Fergus Horrobin          & \hspace{1in} fergus.horrobin@mail.utoronto.ca \\
 35 Chester Ave.          & \hspace{1in} http://fergus.horrobin.com   \\
 Toronto, ON, M4K 2Z8     & \hspace{1in} Phone: (604) 441-8511 \\
\end{tabular}


\vspace{1em}


  \section{EDUCATION}
    \textbf{Bachelor of Science, Specialist, Physical and Mathematical Sciences} \\ 
    University of Toronto Scarborough, Toronto, ON \\
    Minor: Computer Science \\
    Expected to graduate May 2020 \\
    CGPA: 3.97

  \section{TECHNICAL SKILLS}
    Proficient in programming Fortran, C, C++ for parallel computing on CPU, GPU or Coprocessors (MICS).
    Developped a high performance parallel NBody code for the Intel Xeon Phi platform in Fortran with OpenMP and MPI.
    Working knowledge of scientific software such as Python, IDL and Matlab. Some experience with hardware
    design and assembly programming. Experience building and maintaining HPC cluster.

  \section{AWARDS}
   \begin{itemize}
      \item UTSC Academic Travel Fund, November 2017
      \item Center for Planetary Sciences Undergraduate Research Fellowship, May 2017
    \end{itemize}
 
    \section{CONFERENCES}
  \textbf{Numerical Methods for Planet Disk Interactions (NUMPDI)} \\ 
  UNAM, Cuernavaca, MX, Nov 2017 \\ [7pt]
  \textbf{Title:} Type 3 Planet Migration Sutdied with Large Scale NBody Simulations \\ [7pt]
  Results and analysis of Type III planet migration based on simulations performed with a highly parallel NBody code I developped. I show the main features of this type of migration and how it differs from a fast Type I migration. \\ [7pt]
  \textbf{Toronto Meeting on Numerical Integration Methods in Planetary Science} \\
  UTSC, Aug 2017 \\[7pt]
  \textbf{Title:} Numerical Simulations of Planet Migrations on CPU, MIC and GPU \\ [7pt]
  Presented on high performance, parallel N-Body algorithm I developped for studying planet migration. Also presented preliminary results from migration simulations. \\ [7pt]

  \section{RESEARCH EXPERIENCE}
  \textbf{Center for Planetary Science Undergraduate Research Fellow} \hfill Summer 2017 \\ [7pt]
    Working with Prof Pawel Artymowicz from the department of Physics and Astrophysics at UTSC on a research project 
    on orbital mechanics and planet migration. I developped a high performance numerical simulation in Fortran, 
    optimized to run on the Intel Xeon Phi platform to be able to test disks containing as many as 5 billion particles. 
    Through numerical simulations and mathematical models, I have developped some additional mechanisms for 
    the theory of Type III planetary migration.

  \section{TEACHING EXPERIENCE}
  \textbf{Teaching Assistant: CSCA20 Introduction to Programming} \hfill Fall 2017 \\ [7pt]
    Facilitated 2 hour tutorial sessions each week for introductory programming course.
    Weekly activities consisted of additional lecture material and worked problem sets.
    Gained experience working with students of diverse background skill levels.

  \textbf{Introduction to Scientific Programming Seminars} \hfill Fall 2017 \\ [7pt]
    Designed and conducted weekly seminars teaching first year physics students
    introductory Python programming skills. Covered the basics of Python and scientific
    libraries such as Numpy, Scipy and Matplotlib as well as how to approach interesting
    problems from physics ranging from simple worked examples based on kinematics to
    NBody simulations of the solar system. 

  \textbf{Physicd Aid Center (PAC) Tutor} \hfill Fall 2017 \\ [7pt]
    Helped first year stidents in PHYA10 and PHYA11 work through problems.
    Was available 3 hours per week throughout the semester.

  \textbf{Tutoring Introductory Physics and Chemistry} \hfill Summer 2017 \\ [7pt]
    Provided approximately 2 hours per week of tutoring service to 2 students at UTSC
    in PHYA21 and CHMA11. Helped to develope necessary mathematical background and problem
    solving skills to succeed in the courses.

  \section{REFERENCES}
  \textbf{Professor Pawel Artymowicz} \\
  Professor of Physcis and Astrophysics \\ [7pt]
  Research Supervisor for Summer 2017 Research Fellowship \\
  Department of Physical and Environmental Sciences \\
  University of Toronto Scarborough, Toronto, ON \\
  pawel@utsc.utoronto.ca | 416-287-7244 (Work)

  \textbf{Professor Anna Bretscher} \\
  Professor of Computer Science \\ [7pt]
  Course instructor for CSCA20 \\
  Department of Computer and Mathematical Sciences \\
  University of Toronto Scarborough, Toronto, ON \\
  bretscher@utsc.utoronto.ca

\end{resume}
\end{document}
